% \iffalse
\let\negmedspace\undefined
\let\negthickspace\undefined
\documentclass[journal,12pt,twocolumn]{IEEEtran}
\usepackage{cite}
\usepackage{amsmath,amssymb,amsfonts,amsthm}
\usepackage{algorithmic}
\usepackage{graphicx}
\usepackage{textcomp}
\usepackage{xcolor}
\usepackage{txfonts}
\usepackage{listings}
\usepackage{enumitem}
\usepackage{mathtools}
\usepackage{gensymb}
\usepackage{comment}
\usepackage[breaklinks=true]{hyperref}
\usepackage{tkz-euclide} 
\usepackage{listings}
\usepackage{gvv}  
\usepackage{tikz}
\usepackage{circuitikz} 
\usepackage{caption}

\def\inputGnumericTable{}                                
\usepackage[latin1]{inputenc}                 
\usepackage{color}                            
\usepackage{array}                            
\usepackage{longtable}                        
\usepackage{calc}                            
\usepackage{multirow}                      
\usepackage{hhline}                           
\usepackage{ifthen}                          
\usepackage{lscape}
\usepackage{amsmath}
\newtheorem{theorem}{Theorem}[section]
\newtheorem{problem}{Problem}
\newtheorem{proposition}{Proposition}[section]
\newtheorem{lemma}{Lemma}[section]
\newtheorem{corollary}[theorem]{Corollary}
\newtheorem{example}{Example}[section]
\newtheorem{definition}[problem]{Definition}
\newcommand{\BEQA}{\begin{eqnarray}}
\newcommand{\EEQA}{\end{eqnarray}}
\newcommand{\define}{\stackrel{\triangle}{=}}
\theoremstyle{remark}
\newtheorem{rem}{Remark}

\begin{document}
\title{Gate Assignment CH 31}
\author{Shravya Kantayapalam\\ EE23BTECH11030}
\maketitle

\begin{enumerate}
    \item \textbf{Question }:
The position \( x(t) \) of a particle, at constant \( \omega \), is described by the equation
\[
\frac{{d^2x}}{{dt^2}} = -\omega^2 x.
\]
The initial conditions are \( x(t=0) = 1 \) and \( \frac{{dx}}{{dt}}\bigg|_{t=0} = 0 \). 

The position of the particle at \( t = \frac{{3\pi}}{{\omega}} \) is \underline{\hspace{2cm}} (in integer).
\hfill{(GATE CH 2023)}

\solution

\begin{table}[htbp]
    \centering
    \caption{Input Parameters}
    \begin{tabular}{|c|c|}
        \hline
        \textbf{Parameter} & \textbf{Description} \\
        \hline
        $s$ & Complex frequency variable in Laplace domain \\
        $\omega$ & Angular frequency \\
        $X(s)$ & Laplace transform of the function $x(t)$ \\
        $x(t)$ & Time-domain function \\
        \hline
    \end{tabular}
\end{table}

Given:
\begin{align*}
X(s) &= \frac{s}{s^2 + \omega^2} = \frac{A}{s} + \frac{Bs + C}{s^2 + \omega^2}
\end{align*}

Multiplying both sides by \( s^2 + \omega^2 \) yields:
\begin{align*}
s &= A(s^2 + \omega^2) + (Bs + C) \cdot s \\
s &= As^2 + A\omega^2 + Bs^2 + Cs \\
(A + B)s^2 + Cs + A\omega^2 &= s
\end{align*}

This equation must hold for all values of \( s \). Therefore, the coefficients must match term by term.
\begin{align*}
A\omega^2 &= 0 \implies A = 0 \\
\text{For the } s \text{ terms:} &\quad C = 1 \\
\text{For the } s^2 \text{ terms:} &\quad A + B = 0 \implies B = -A = 0
\end{align*}

Thus, \( A = 0 \), \( B = 0 \), and \( C = 1 \).
Therefore:
\begin{align*}
X(s) &= \frac{s}{s^2 + \omega^2} = \frac{1}{s^2 + \omega^2}
\end{align*}

The inverse Laplace transform of \( X(s) \) is:
\begin{align*}
x(t) &= \mathcal{L}^{-1} \left\{ \frac{1}{s^2 + \omega^2} \right\} = \sin(\omega t)
\end{align*}

\end{document}

